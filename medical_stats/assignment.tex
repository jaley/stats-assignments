\documentclass{article}

\usepackage{listings}
\usepackage{hyperref}

\title{Medical Statistics Assignment}
\author{James Aley}
\date{November 2015}

\lstset{ %
  breakatwhitespace = false,
  breaklines = true,
  tabsize=2,
  showspaces=false
}

\begin{document}

\maketitle

\section{Analysis of a Journal Article}

This article discusses statistical elements of the analyses in a
medical study focusing on the impact coffee consumption may have on
risk of individuals later suffering from type 2 diabetes mellitus
\cite{vandam}.

The study is a prospective analysis, using individuals
selected randomly from public registers in two Dutch
towns. Participants were sent a questionnaire to record various
aspects of their lifestyle considered relevant to the
study. Importantly, this of course covered how much coffee the
participants consumed, measured in cups/day, but also various other
factors that were thought to be possibly confounding in the study.

We note at this point that this is necessary, as though the study is
prospective, it is \emph{not} a clinical trial. We do have an element
of randomization to help control for confounds, but the explanatory
variable of interest, the  amount of coffee consumed by each
individual, has not been actively manipulated as a part of the
study. The study therefore can only hope to demonstrate a strong
association between coffee consumption and risk of type 2 diabetes. We
will see later, however, that some considerable effort is made to
monitor the impact of potentially confounding factors.

The study excludes a number of individuals upfront as part of the
design. This includes people that are under 30 years old, pregnant, or
had missing information in the survey. After these exclusions, the
total number of participants was 17,111. These individuals were sent
follow-up questionnaires over subsequent years to check whether they
had been diagnosed with type 2 diabetes. In total, 306 such cases were
recorded in the study.

A 95\% confidence interval for incidence of type 2 diabetes within
this population would therefore be (0.016, 0.020), with estimated
value 0.017. Other studies placed overall population incidence for the
Netherlands at 2.2\% in 2003 \cite{prevalence}. It seems likely that
the age restrictions on this study (30-60 years) may explain this
slightly lower observation in the study cohort. However, we would also
expect there to be some selection bias, in that participants diagnosed
with diabetes may be more enthusiastic to respond to follow-ups than
others.

The article covers some measures used to monitor and control for
potential confounds in the analysis. This seems like a good idea, as
coffee consumption is likely to correlate quite strongly with various
other lifestyle choices that could have an impact. Many such potential
confounding factors are identified in the study, but we should note
that this highlights one of the main drawbacks in this kind of
study. In order to test for confounds, we have to propose and
enumerate them. It is entirely plausible that others exist, that we're
unable to foresee at the beginning of the study, when designing the
questionnaire for participants to complete. We therefore would have no
data about such factors and be unable to control for them when
interpreting the results.

The article states that for each of the characteristics they measured
in participants, tests for correlation with coffee consumption were
``obtained by modeling the median value of each category of coffee
consumption as a continuous variable''. It's not entirely clear what
is meant by this, as no further details are provided in other parts of
the article, however the need to map consumption categories to a
continuous variable perhaps suggests some kind of regression analysis
is being used.

These tests find that coffee consumption correlates
somewhat with all characteristics except for cardiovascular disease
and consumption of dark bread. One of the more surprising elements of
this study is that those characteristics showing positive correlation
with coffee consumption are generally considered to have a negative
impact on health. If we were to attribute the conclusions in this
study to the confounding factors considered here, it would perhaps be
even more surprising than attributing it to coffee consumption anyway.

The study makes use of the \emph{Cox proportional hazards model}
\cite{cox} to estimate relative risk for participants as a function of
their coffee consumption, time and other factors suspected to be
confounding. This is a time-series analysis technique, which seems
like a sensible choice, given that we cannot ignore how time plays an
important role in the diagnosis of diseases such as type 2
diabetes. We do not have an immediately measurable response variable;
all we know is that a participant has or has not \emph{yet} been
diagnosed. Censoring therefore plays an important role in interpreting
the results.

The Cox proportional hazards model relies on the \emph{proportional
  hazards assumption}, given below:

\begin{equation}
h_{z}(t) = g(z)h_{0}(t)
\end{equation}

Where \(z\) is a vector of explanatory variables, \(h_{0}(t)\) is the
baseline or nominal hazard function and \(g(z)\) is a function of
\emph{only} the explanatory variables (not time), which we multiply by
the baseline hazard rate to obtain the hazard function taking \(z\)
into account. Thus we assume that the impact of \(z\) does not vary
with time beyond the base hazard rate.

It's very hard to check whether this assumption is reasonable, as we
don't know what (if anything) in coffee could be helping to prevent
type 2 diabetes. We certainly have no information to tell us whether
such an effect will vary with time, whether it be how long a
participant has been drinking coffee at the specified quantity, their
age or length of exposure to potential causes.

Proceeding with the Cox regression model, however, we can reasonably
assume that the relative risk estimates cited in the article have been
produced using a model of the form:

\begin{equation}
\frac{h_{z}(t | X_1, \ldots, X_k)}{h_0(t)} = \exp(\beta_1 X_1 + \ldots +
\beta_k X_k)
\end{equation}

The ratio of \(h_z\) to \(h_0\) would be the relative risk reported,
and the \(X_1 \ldots X_k\) are explanatory variables for the
regression model. In this case, the explanatory variable of primary
interest would be coffee consumption levels. Again, the article
mentions that consumption categories were mapped to continuous
variables with the same technique as before.

The conclusions reached about the relative risk for high consumers of
coffee being approximately 0.50 are based on inferences around the
corresponding \(\beta\) value in the regression model. If we let
\(\beta_c\) be our coefficient corresponding to the continuous random
variable created for coffee consumption in cups/day, then the p-values
provided are presumably testing the null hypothesis \(H_0: \beta_c = 0
\) vs the alternative \(H_1: \beta_c \neq 0\). The very low \textit{p}-value
indicates that level of coffee consumption significantly improves the
model for predicting diagnoses over time.

It is useful to consider at this point what the relative risk, or
hazard ratio in this model actually represents, and how the response
variable was initially measured in the study. The hazard function
itself is the probability of an event occurring after time \(t\),
\emph{in the next instant}. In this study, that probability represents
diagnosis of type 2 diabetes. However, such a diagnosis is not really
an immediate effect, it requires that a participant notices symptoms
in themselves, visits a medical professional and then later completes
a questionnaire providing the date this happened. Many patients first
go through phases of being ``pre-diabetic'' before their conditions
worsen and a diagnosis of full type 2 diabetes is issued.

The requirement for action from the patient in order to give us a
value for \(t\) could pose problems. It was already demonstrated in
the analysis of potential confounding factors that coffee consumption
correlates quite strongly with various other (mostly unhealthy)
life-style choices. It seems entirely possible that participants
drinking more coffee may also visits doctors less frequently, on
average. They may also be less likely to perceive problems in the
symptoms, as they are initially quite subtle (increased thirst
and hunger, for example). The study makes no mention of this, or of
any requirement for participants in the study to regular have
checkups, which may have helped control for issues like this.

Though we can see that there are various parts of the study
susceptible to systematic problems, the overall conclusions are
certainly interesting. We can see that there are a few points where
the statistical analysis may have been somewhat sub-optimal, but for such a
strong measured effect, it seems unlikely that it would impact
conclusions substantially. The biggest risk seems to be that perhaps
the very design of the study, relying on action from participants to
provide a value for the time until diagnosis, may actually correlate
with the same personality traits that the study demonstrated coffee
consumption does. The methods used to control for confounds in this
study do not control for that particular source of potential bias. It
may be interesting to repeat the analysis with logistic regression,
modeling the proportion of cases in a fixed amount of time, rather
than as a function of time. If the hypothesis about diagnosis time
correlating with consumption is correct, we would expect to see a
smaller effect this way.

\section{Newton-Raphson for Binomial Max-Likelihood}

The code in the following listing provides an implementation for the
Newton-Raphson algorithm, to fit parameters that maximize the
likelihood function for the Binomial distribution. The code can also
be access online, see \cite{github}.

\lstinputlisting[language=R]{assignment_newton_raphson.R}

\begin{thebibliography}{9}

\bibitem{vandam}
Rob M van Dam, Edith J M Feskens.
\textit{Coffee Consumption and risk of type 2 diabetes mellitus.}
Lancet 2002; 360: 1477–78

\bibitem{prevalence}
Department of Family Practice, University of Groningen.
\emph{Prevalence, incidence and mortality of type 2 diabetes mellitus
revisited: a prospective population-based study in The Netherlands
(ZODIAC-1).}
Eur J Epidemiol. 2003;18(8):793-800.

\bibitem{cox}
Cox, D. R., and Oakes, D. (1984).
\emph{Analysis of Survival Data}.
Chapman and Hall, London, New York.

\bibitem{github}
James Aley on GitHub.
\emph{Newton-Raphson assignment source code.}
\url{https://github.com/jaley/stats-assignments/blob/master/medical_stats/assignment_newton_raphson.R}

\end{thebibliography}


\end{document}

%%% Local Variables:
%%% mode: latex
%%% TeX-master: t
%%% End:
