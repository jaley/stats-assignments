%%% Local Variables:
%%% mode: latex
%%% TeX-master: t
%%% End:

\documentclass{article}

\usepackage{amsmath}
\usepackage{amssymb}
\usepackage{minted}
\usepackage{graphicx}

\graphicspath{ {images/} }

\title{Modern Statistical Methods Assignment}
\author{James Aley}
\date{February 2016}

\begin{document}

\maketitle

\section*{Question 1}

\subsection*{(a)}

The Pareto distribution has probability density function $f(x)$ given by:

\begin{displaymath}
  f(x) = \frac{\alpha \beta^{\alpha}}{{(x + \beta)}^{\alpha + 1}} \quad x \geq 0
\end{displaymath}

To calculate the cumulative distribution function, we integrate this
function over the range $\left[ 0, u \right]$:

\begin{align*}
  F(u) &= \int_0^u f(x) \, \mathrm{d}x \\
       &= \int_0^u \frac{\alpha \beta^{\alpha}}
                            {{(x + \beta)}^{\alpha + 1}}
         \, \mathrm{d}x \\
       &= {\left[ - \beta^\alpha {(x + \beta)}^{-\alpha}  \right]}_0^u \\
       &= 1 - \frac{\beta^{\alpha}}{{(u + \beta)}^\alpha} \\
       &= 1 - {\left( \frac{\beta}{u + \beta} \right)}^{\alpha}
\end{align*}

Now, to calculate the expected value, using integration by parts:

\begin{align*}
  \mathbf{E}\left[X\right] &= \int_0^{\infty} x f(x) \mathrm{d}x \\
                           &= \int_0^{\infty} \frac{x \alpha \beta^{\alpha}}
                             {{(x + \beta)}^{\alpha + 1}}
                             \, \mathrm{d}x \\
                           &= {\left[
                             - \beta^\alpha x {(x + \beta)}^{-\alpha}
                             + \int \beta^{\alpha} {(x + \beta)}^{- \alpha}
                             \right]}_0^{\infty} \\
                           &= {\left[
                             - \beta^\alpha x {(x + \beta)}^{-\alpha}
                             + \frac{\beta^{\alpha}}{1 - \alpha}
                             {\left( x + \beta \right)}^{1 - \alpha}
                             \right]}_0^{\infty} \\
\end{align*}

We can see by taking limits on the upper bounds, that we require
$\alpha > 1$ in order for this integral to converge.

\begin{align*}
  \lim_{x \to \infty} {\left[
  - \beta^\alpha x {(x + \beta)}^{-\alpha}
  + \frac{\beta^{\alpha}}{1 - \alpha}
  {\left( x + \beta \right)}^{1 - \alpha}
  \right]}_0^{\infty} = 0
  \quad \left(  \alpha > 1 \right)
\end{align*}

Therefore, substituting in $x = 0$ and subtracting from $0$ gives us:

\begin{align*}
  \mathbf{E}\left[X\right] &= 0 - \frac{\beta^\alpha}{\beta^\alpha}
                             \left( \frac{\beta}{1 - \alpha}
                             \right) \\
                           &= \frac{\beta}{\alpha - 1}
\end{align*}

For the median, we need to find the value $m$ that puts equal
probability mass on either side of it. That is to say that $F(m) =
\frac{1}{2}$

\begin{align*}
  F(m) &= \frac{1}{2} \\
  1 - {\left( \frac{\beta}{m + \beta} \right)}^{\alpha} &= \frac{1}{2} \\
  \frac{\beta}{m + \beta} &= {\left( \frac{1}{2} \right)}^{\frac{1}{\alpha}} \\
  m &= \beta \left( \frac{1}{2^{\frac{1}{\alpha}}} - 1 \right)
\end{align*}

To calculate the variance, $\mathbf{Var}\left[ X \right]$, we shall
make use of the formula:

\[
  \mathbf{Var}\left[ X \right] = \mathbf{E}\left[X^2\right]
  - {\mathbf{E}\left[X\right]}^2
\]

We already have the first moment available from from calculating the
mean earlier, so now we need to calculate the second moment,
$\mathbf{E}\left[X^2\right]$ as follows. This time, we'll need to
apply integration by parts twice successively.

\begin{align*}
  \mathbf{E}\left[X^2\right] &= \int_0^{\infty}
                               x^2 f(x) \, \mathrm{d}x \\
                             &= \int_0^{\infty}
                               \frac{x^2 \alpha \beta^\alpha}
                               {{\left( x + \beta \right)}^{\alpha +
                               1}} \, \mathrm{d}x \\
                             &= \alpha \beta^\alpha
                               {\left[
                               \frac{-x^2
                               {\left( x + \beta \right)}^{- \alpha}}
                               {\alpha}
                               - \int \frac{-2x{(x + \beta)}^{- \alpha}}
                               {\alpha}
                               \right]}_0^{\infty} \\
                             &= \beta^\alpha
                               {\left[
                               -x^2 {(x + \beta)}^{-\alpha}
                               + \left(
                               \frac{2x{\left(x + \beta \right)}^{1 - \alpha}}
                               {(1 - \alpha)}
                               - \int \frac{2{\left( x + \beta
                               \right)}^{1 - \alpha}}
                               {(1 - \alpha)}
                               \right)
                               \right]}_0^{\infty} \\
                             &= \beta^\alpha
                               {\left[
                               -x^2 {(x + \beta)}^{-\alpha}
                               + \frac{2x{\left(x + \beta \right)}^{1 - \alpha}}
                                 {(1 - \alpha)}
                               + \frac{2 {\left(x + \beta \right)}^{2 - \alpha}}
                                 {(1 - \alpha)(2 - \alpha)}
                               \right]}_0^{\infty}
\end{align*}

This integral will only converge if we require that $\alpha > 2$, in
which case we take limits for $x$ as before:

\begin{displaymath}
\lim_{x \to \infty}
\beta^\alpha
  {\left[
  -x^2 {(x + \beta)}^{-\alpha}
  + \frac{2x{\left(x + \beta \right)}^{1 - \alpha}}
  {(1 - \alpha)}
  + \frac{2 {\left(x + \beta \right)}^{2 - \alpha}}
  {(1 - \alpha)(2 - \alpha)}
  \right]}
= 0
\end{displaymath}

Note that the requirement that $\alpha > 2$ is important both so that
the $x^2$ component above grows more slowly than the denominator, and
so that the final quotient is finite. Again, we substitute in $x = 0$
and subtract from the $0$ obtained from this limit to get an
expression for the second moment:

\begin{align*}
  \mathbf{E}\left[X^2\right] &= 0 - \beta^{\alpha} \left(
                               \frac{2 \beta^{2 - \alpha}}
                               {(1 - \alpha)(2 - \alpha)}
                               \right) \\
                             &= \frac{2 \beta^2}
                               {(\alpha - 1)(\alpha - 2)}
\end{align*}

Returning to the expression for $\mathbf{Var}\left[ X \right]$, we have:

\begin{align*}
  \mathbf{Var}\left[ X \right] &= \mathbf{E}\left[X^2\right]
                                 - {\mathbf{E}\left[X\right]}^2 \\
                               &= \frac{2 \beta^2}
                                 {(\alpha - 1)(\alpha - 2)}
                                 - {\left(
                                 \frac{\beta}{\alpha - 1}
                                 \right)}^2 \\
                               &= \frac{2 \beta^2 \alpha (\alpha - 1)
                                 - \beta^2 (\alpha - 2)}
                                 {{(\alpha - 1)}^2(\alpha - 2)} \\
                               &= \frac{\beta^2 \alpha}
                                 {{(\alpha - 1)}^2(\alpha - 2)}
\end{align*}

\subsection*{(b)}

To generate samples from the Pareto distribution, using uniform
variables, we can invert the CDF obtained in the previous section as
follows.

\begin{align*}
  F(x) &= 1 - {\left( \frac{\beta}{x + \beta} \right)}^\alpha \\
  \frac{\beta}{x + \beta} &= {\left( 1 - F(x) \right)}^{\frac{1}{\alpha}} \\
                        x &= \beta \left[{ \left( 1 - F(x)
                            \right)}^{-\frac{1}{\alpha}} - 1 \right]
\end{align*}

This gives us an expression for $F^{-1}(x)$, the inverse cumulative
distribution function:

\[
  F^{-1}(u) = \beta \left[{ \left( 1 - u \right)}^{-\frac{1}{\alpha}}
    - 1 \right]
\]

We may now use values for $u$, which should be realisations of  $U
\sim \mathrm{Uniform}(0, 1)$. These may be generated using a
congruential generator, or any standard method for generating
pseudo-random uniform numbers. The \texttt{runif()} function can be
used to achieve this in R.

\subsection*{Note: R Plotting Requirements}

The remainder of this document makes use of the \texttt{ggplot2} R
package for rendering plots. The following snippet of code can be run
to make this available in order to reproduce the plots:

\begin{minted}{R}
# Install ggplot if it's not already available
install.packages('ggplot2')

# Now load the functions from the ggplot2 package
require(ggplot2)
\end{minted}

\subsection*{(c)}

The following code snippet includes implementations of the Pareto
distribution PDF, CDF and inverse CDF, as calculated in the previous
section. Following those function definitions, we make use of them to
simulate values from the Pareto distribution and visualise that
simulation using the method outlined previously.

\begin{minted}{R}
#' Probability density function for the Paretro distribution, with
#' parameters alpha and beta
pareto.pdf <- function(alpha, beta, x) {
  (alpha * beta^alpha) / (x + beta)^(alpha + 1)
}

#' Cumulative distribution funtion for Paretro distribution with paremeters
#' alpha abd beta.
pareto.cdf <- function(alpha, beta, u) {
  1 - (beta / (u + beta))^alpha
}

#' Inverse cumulative distribution function for the Pareto
#' distribution. Alpha and Beta are distribution parameters,
#' q should be a value between 0 and 1.
pareto.inv.cdf <- function(alpha, beta, q) {
  beta * ((1 - q)^(-1/alpha) - 1)
}

## Compare simulated values to the density function
pareto <- data.frame(simulated = pareto.inv.cdf(3, 100000, runif(10000)))
ggplot(pareto, aes(simulated)) +
  geom_density(colour="red", fill="red", alpha=0.1) +
  stat_function(fun = function(x) {pareto.pdf(3, 100000, x)},
                colour="blue", geom="area", alpha=0.1, fill="blue") +
  xlab("X") +
  ylab("Density") +
  xlim(0, 100000)
\end{minted}

Note that in Figure \ref{fig:q1c_histogram} we're plotting the real
Pareto PDF against the \emph{density} of the simulated values, so that
they are on comparable scales. We're also focusing on the range for
$X$ in 0 to 100,000, before entering the very long tail. We can see
that the simulation fits the PDF reasonably well, and is mostly unable
to reach the very low extremes of the real PDF.

\begin{figure}
  \includegraphics[width=\textwidth]{q1c_histogram}
  \caption{Simulated value density (red) plotted against exact Pareto
    PDF (blue).}
  \centering
  \label{fig:q1c_histogram}
\end{figure}

\subsection*{(d)}

Insurance claims are likely to me well-modeled by the Pareto
distribution, as it is very heavy-tailed. It is able to realistically
represent the relatively low-probability large claims, corresponding to
extreme loss for the insurer. Insurance claims are also likely to
follow the \emph{Pareto Principle}, in that most of the value of the
payments made will be to relatively few (large) claims.

\section*{Question 2}

The following R code snippet simulates possible outcomes for 1000
customers, each with a claim probability of 0.1 in a year. We assume
the claims $X$ have distriubtion $X \sim \mathrm{Pareto}(3, 100000)$
as before.

We also note, given the restriction that each customer can only make a
single claim in a year, we essentially have a Bernoulli outcome variable
for whether or not each customer makes a claim with parameter $p =
0.1$. Modeling the event of a claim occurring as $1$ and not occurring
as $0$, we can extend this to $n$ customers be using a Binomial
distribution. If we let $N$ represent the number of claims, then:

\[
  N \sim \mathrm{Bin}(1000, 0.1)
\]

The code below uses this to model the number of claims,
\texttt{n\_claims} in each simulation iteration. That variable tells us
the number of uniform variables to draw and pass on to our inverse
Pareto CDF derived previously to model the value of claims for one year.

\begin{minted}{R}
#' Returns a vector of length n_runs containing simulated values
#' for the year end assets.
simulate.year.end.assets <- function(n_runs = 10000,
                                     start_assets = 250000,
                                     n_customers = 1000,
                                     premium = 6000,
                                     claim_prob = 0.1) {
  # Accumulate year-end assets in a vector
  outcomes <- numeric(length = n_runs)
  for(i in 1:n_runs) {
    # Number of claims this year, in this simulation
    n_claims <- rbinom(1, n_customers, claim_prob)

    # Calculate year-end assets for this number of claims, by
    # drawing from Pareto distribution
    outcomes[i] <- start_assets +
      (n_customers * premium) -
      sum(pareto.inv.cdf(3, 100000, runif(n_claims)))
  }
  return(outcomes)
}

# Load simulation results into a data frame for plotting
# a histogram of the outcomes
bankruptcy <- data.frame(assets=simulate.year.end.assets())
ggplot(bankruptcy, aes(assets)) +
  geom_density(colour="blue", fill="blue", alpha=0.1) +
  geom_vline(xintercept = 0, linetype="longdash") +
  xlab("Assets at Year End") +
  ylab("Probability Density")

# Calculate the probability of bankruptcy, i.e. the proportion
# of outcomes where year end profits were less than zero.
sum(bankruptcy['assets'] < 0) / n_runs
\end{minted}

In one simulation, we see a probability of bankruptcy probability of
0.0954, from the outcome distribution shown in Figure
\ref{fig:q2_histogram}.

\begin{figure}
  \includegraphics[width=\textwidth]{q2_histogram}
  \caption{Simulated year-end profit or loss.}
  \centering
\label{fig:q2_histogram}
\end{figure}

\section*{Question 3}

\subsection*{(a)}

The following code snippet makes use of the simulation function we
defined in the previous section, calling it with a different premium
price parameter many times to produce the plot in Figure~\ref{fig:q3_premiums}.

\begin{minted}{R}
#' Esimate the probability of bankruptcy for a premium of given price
bankruptcy.prob <- function(premium) {
  N <- 10000
  assets <- simulate.year.end.assets(premium=premium, n_runs=N)
  sum(assets < 0) / N
}

premiums <- seq(from=5500, to=8000, by=250)
premium.data <- data.frame(premium.price = premiums,
                           bankruptcy.prob = sapply(premiums, bankruptcy.prob))


ggplot(premium.data, aes(x=premium.price, y=bankruptcy.prob)) +
  geom_line(colour="blue") +
  stat_hline(yintercept = 0.02, linetype="longdash") +
  xlab("Premium Price (GBP)") +
  ylab("Probability of Bankruptcy")
\end{minted}

\begin{figure}
  \includegraphics[width=\textwidth]{q3_premiums}
  \caption{Probability of bankruptcy for changing premium prices.}
  \centering
\label{fig:q3_premiums}
\end{figure}

As indicated by the dashed line in Figure~\ref{fig:q3_premiums}, we
should set the premium to \pounds{7000} or higher in order to see a probability
of bankruptcy below 2\%.

\subsection*{(b)}

Again, making use of the simulation functions defined previously, the
following code snippet reruns the simulation for varying claim
probabilities to produce the plot shown in Figure~\ref{fig:q3_claims}.

\begin{minted}{R}
bankruptcy.prob.claim <- function(claim.prob) {
  N <- 10000
  assets <- simulate.year.end.assets(claim_prob = claim.prob, n_runs=N)
  sum(assets < 0) / N
}

claims <- seq(from=0.05, to=0.15, by=0.005)
claim.data <- data.frame(claim.prob = claims,
                         bankruptcy.prob =
                           sapply(claims, bankruptcy.prob.claim))

ggplot(claim.data, aes(x=claim.prob, y=bankruptcy.prob)) +
  geom_line(colour="blue") +
  stat_hline(yintercept = 0.02, linetype="longdash") +
  xlab("Claim Probability") +
  ylab("Probability of Bankruptcy")
\end{minted}

Holding the premium price fixed at \pounds{6000}, as per the original
specification, we can see from Figure~\ref{fig:q3_claims} that the
company can only expect a bankruptcy probability below 2\% if the
claim probability is below 0.08.

\begin{figure}
  \includegraphics[width=\textwidth]{q3_claims}
  \caption{Bankruptcy probability for varying customer claim
    probabilities.}
  \centering
\label{fig:q3_claims}
\end{figure}

\section*{Question 4}

This report summarises an investigation into company year-end asset
values and provides recommendations accordingly. The investigation in
question involved running simulations based on information we
currently hold on customers and insurance claim behaviour. The
following details have been assumed to be accurate throughout
simulations:

\begin{itemize}
  \item Number of customers: 1,000
  \item Annual premium charged: \pounds{6,000}
  \item Current asset value: \pounds{250,000}
  \item Customers can only make one claim per year
  \item 10\% probability of a customer making a claim
  \item Claims prices can be reasonably modeled by a Pareto distribution with
    parameters $\alpha = 3$ and $\beta = 100,000$
\end{itemize}

Based on these assumptions, the mean asset valuation at the end of the
year in simulations was approximately \pounds{1250000}. However,
variability was seen to be very high, and in fact a 95\% confidence
interval for asset valuation was (-\pounds{900000},
\pounds{2900000}). That is to say that though substantial profits
are entirely likely, we cannot rule out the possibility of a
significant loss or bankruptcy at the 95\% level of significance.

In simulations, we found that there is currently about a 10\% chance
of bankruptcy with current pricing and claim characteristics. This is
worryingly high, and certainly something that the company should look
to address in order to secure sustainable business. Various factors
have been investigated in simulations to assess impact on these
metrics and are summarised in following sections.

\subsection*{Premium Price}

Unsurprisingly, increasing the customer premium price can be shown to
reduce probability of bankruptcy and increase predicted profits
according to simulations. Figure~\ref{fig:q3_premiums} shows how
increases in the current premium price reduce the probability of
bankruptcy. If we work with a target of reducing the probability to
2\%, indicated by the dotted line, then increasing the premium price
to \pounds{7000} would be sufficient.

The corresponding mean profit simulated was \pounds{2250000} with 95\%
confidence limits of (\pounds{166000}, \pounds{3900000}). It should of
course be noted however that this change would likely cause some
customers to revisit their choice of insurance supplier, and we would
inevitably lose some customers. This impact of the number of customers
buying insurance is investigated in a subsequent section.

\subsection*{Claim Probability}

The probability of customers making a claim has a clear impact on
profits and risk of bankruptcy too, as shown in
Figure~\ref{fig:q3_claims}. Reducing the probability of a claim from
the current 10\% to 8\% would also reach the 2\% bankruptcy risk
target. Impact on profit estimates are comparable to the suggest
increase in premium price from the previous section.

It should be noted, however, that this factor is likely more difficult
for the company to control. Various measures could be taken to limit
what customers are able to make claims for, but again this is likely
to cause customers to choose other suppliers in some circumstances.

\subsection*{Number of Customers}

Perhaps the least risky option in terms of unpredictable and
potentially undesirable consequences would be to increase the number
of customers buying insurance. Figure~\ref{fig:q4_customers}
illustrates the impact of this initiative on bankruptcy risk via
simulations again.

\begin{figure}
  \includegraphics[width=\textwidth]{q4_customers}
  \caption{Bankruptcy probability for increasing customer numbers}
  \centering
\label{fig:q4_customers}
\end{figure}

The 2\% target can be reach by increasing our customer base four times
to 4,000. If it is possible to achieve this through advertising
initiatives then we could substantially improve profits and decrease
bankruptcy risk without putting our existing customer relationships at
risk.

\subsection*{Recommendations}

Any of the changes previously mentioned for the above factors could
improve on the current situation, but some difficult judgment calls
need to be made on the trade-offs we expect to see. A few key pieces
of information that would help in this respect are:

\begin{itemize}
  \item How many customers do we expect will cancel their policy per
    unit increase in premium price?
  \item Are there are any policy changes that can be made to reduce
    the probability of a claim from each customer, and again, what
    would the impact be in terms of cancellations?
  \item What would the cost be of advertising initiatives required to
    increase customer base to the suggested levels?
\end{itemize}

With this information, further modeling could be carried out to
provide more reliable suggestions. However, in the absence of this
research it seems that it might be a good idea to hedge bets by
instead designing further insurance products (assuming they can reach
the market in a reasonable time-frame) that exhibit one of the following:

\begin{itemize}
  \item Attract much higher customer volumes with lower claim
    probabilities, likely at the cost of a lower premium price. That
    is, an ``economy'' or ``entry-level'' policy for a new class of
    customer. Equivalently, we could design this policy to have a cap
    on the maximum payout or a more predictable distribution of claims
    to limit losses.
  \item A ``high-end'' option with a much higher premium, again
    targeting new customers. We expect claims with higher probability
    here and offer a high cap on maximum claim, but cover this with a
    bigger premium.
\end{itemize}

\end{document}
